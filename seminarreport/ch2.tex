\chapter{How Cloud Computing Can Benefit From NETINF}
Cloud computing is today offering an efficient
environment for quick deployment of new services. In
particular, it offers unrivalled opportunities for quickly
scaling up the capacity for services that suddenly become
popular. Infrastructure as a Service (IaaS) and Platform as a
Service (PaaS) approaches have proven to be effective
means for separating the service deployment from the
provisioning of the required service infrastructure. The latter
obviously is a much slower process as it includes hardware
installation and possibly extended physical facilities, as well
as establishment of new communication infrastructure. One
of the major hurdles with current cloud computing is the
management costs of the computing, storage and networking
facilities. To make the management tractable, today’s
approach to cloud computing depends on the site
infrastructure and the networking topology being relatively
stable.
The main benefit of what is presented in this paper is not
a way to improve the cloud computing functionality as such
by use of NetInf. It rather argue that by basing the cloud
computing infrastructure on NetInf, many of today’s issues
with management of cloud computing infrastructures can be
eased.
In this light, distributing cloud computing service objects
across a more dynamic networking environment like moving
networks, or a home networking environment, is a very
difficult task with today’s cloud computing technology. In
addition, cloud computing service objects need to be copied
between servers, e.g. to realize load sharing. When using
existing technologies, such issues would need to be handled
by a combination of more or less automated network
management tools, routing protocols, and add-on protocols
for host and site multihoming, as well as host and network
mobility. This adds substantial complexity to the system, and
the mechanisms may interact with each other as well as with
NATs and firewalls in an unpredictable fashion.

The following paragraphs illustrate how introducing
NetInf technology can help addressing the above cloud
computing challenges.

The Network of Information paradigm allows for a new
approach to the problem of dynamic network topologies.
While cloud computing brings about revolutionary
technology to resource sharing, the Network of Information
is a new approach to accessing information (in its widest
sense) on a network. It brings a new abstraction to the
networking layer: the notion of information objects.
Compared to existing networking paradigms, this new
approach is designed for directly accessing the information,
rather than addressing it indirectly via the host or network
domain containing the information. The location of the
information thereby becomes secondary, which makes it
much easier to deal with configuration changes and mobility.
The addressing used in NetInf fully relies on naming the
information itself, and not the location or network domain it
is retrieved from. For a cloud computing platform, this
significantly simplifies the way data is handled.
This property also allows for hiding reconfigurations of the
supporting nodes and networks from the cloud computing
platform. It could potentially also make cloud computing
deployable on a smaller scale, like home networking, and
even in environments with mobile nodes and networks.

When the access of information objects is no longer done
in an application-specific fashion, integration and
composition of applications become easier, since they all
access information using the same naming scheme. 

The media distribution capabilities of NetInf provide a
part of implementing IaaS. A part of IaaS or PaaS offerings
is often the capability to efficiently distribute large amounts
of content, e.g. video files. This can be realized via content
distribution overlays, such as the well-known Akamai CDN. Content distribution functionality is
provided natively in a NetInf-enabled network, realized by
the functions introduced in the previous section. Information
is simply requested by specifying its name, while a NetInf-enhanced networking layer takes care of choosing the
optimal distribution mechanism, including caching and
source selection. Similar to the popular peer-to-peer systems,
clients can also serve as a new source for already
downloaded information.
NetInf functionality will also support services and service
composition. Without aiming to become a service platform
in itself, the idea is that services can be treated as a type of
information objects. Also a service needs to be
unambiguously identified, and can have multiple instances in
the network. Meta-data can be used to describe the
characteristics of the service, and to support the proper
selection of a service. This applies both to cases where the
selection takes place manually, e.g. by a user browsing a
service registry, and to cases where this happens in an
automated fashion, e.g. in a service composition engine. In
this way many mechanisms such as load balancing and
mobility that are available through NetInf can be reused also
for services.

Through a close cooperation of cloud computing and
networking of information, network nodes and network
resources, including various storage systems, have the
potential of becoming truly transparent to the applications
and the users. The natural evolution of networking is thus to
move from networking of nodes to networking of
information objects. But for users to feel comfortable with
the, initially appealing, idea of just dropping their
information objects into the network, for storage, processing
and distribution, there are a number of issues that need to be
addressed and are currently being worked on in the 4WARD
project. These include security issues like confidentiality,
integrity, privacy and access rights. Also requirements on
reliability and availability will pose challenges. While these
all are first rank challenges, the major hurdle
on the way to networking of information lies in the
scalability issues associated with changing the granularity of
networking from, relatively, few nodes to the extremely
plentiful information objects themselves. 





