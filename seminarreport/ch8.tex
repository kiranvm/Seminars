\chapter{\label{res}Conclusion and Future Work}
The main purpose of this paper is to investigate what
benefits the Networks of Information (NetInf) technology
can bring to the infrastructure that is the foundation for cloud
computing. In particular, the management of a cloud
computing infrastructure can be simplified, as it does not
have to deal with the details of storing and transporting
information objects.

This paper presents the NetInf architecture,
and described how it can support cloud computing services
by offering an API that hides the dynamics of object
locations and network topologies. One single name
resolution and routing mechanism is used, regardless of
whether the dynamics depend on network reconfigurations,
change of service level agreements, mobility events, re-
homing events, or any other type of network event. The task
of designing cloud computing services that are robust against
object re-locations or changes in the topology of the
underlying infrastructure network can thereby be
significantly simplified.

To illustrate the NetInf approach, a novel routing
mechanism based on late locator construction has been
described that performs object-to-object routing rather than
traditional host-to-host routing. This mechanism can operate
over a highly dynamic network topology and allows for
scalable handling of a very large number of objects.

Future work includes more detailed investigations on
how NetInf can handle services, including the use of NetInf
as a service directory for Web Services. Apart from the
features described in this paper, also automated and
distributed processing of information objects are being
investigated, e.g. to offer a delay-sensitive service as close as
possible to the end-user. As both NetInf and virtualization
(Vnet) of network resources are part of a common
architecture being developed in the 4WARD project,
studies are being conducted on which additional benefits their
combination can bring to cloud computing.
